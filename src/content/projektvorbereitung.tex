%! Author = Luca
%! Date = 11.12.2022
Im Vorfeld eines jeden Projektes ist es wichtig, die aktuelle Situation (Ist-Analyse) und die gewünschte Zielsituation (Soll-Analyse) zu analysieren, um die Ziele und Anforderungen des Projektes zu definieren und die notwendigen Ressourcen zu planen.

\subsection{Ist-Analyse}
Der aktuelle Projektstand beinhaltet, dass der Großauftrag der Schäfer & Twachtmann Immobilien GbR noch am Anfang steht und als erster Schritt die gemeinsame Datenbank entwickelt werden soll, um die Grundlage für die zukünftige Terminplanungssoftware und Immobilienverwaltungssoftware zu bilden.

\subsection{Soll-Analyse}
Am Ende des Projektes soll der Entwurf zu der geforderten Datenbank stehen.
Die Anforderungen ergeben sich wie folgt:
\begin{itemize}
    \item Für jede Immobilie soll die Adresse, Wohnfläche, Anzahl der Zimmer, Preis, Einstellungsdatum, Verkaufsdatum und das Baujahr gespeichert werden.
    \item Es soll verschiedene Immobilientypen wie Einfamilienhaus, Mehrfamilienhaus, Reihenmittelhaus, Reihenendhaus etc. geben.
    \item Jede Immobilie soll von einem Makler betreut werden, der über einen uneindeutigen Nachnamen, Vornamen und eine Telefonnummer verfügt.
    \item Die Eigentümer jeder Immobilie sollen mit Vornamen, Nachnamen und Telefonnummer gespeichert werden.
    \item Das Maklerbüro hat Stammkunden, die im Laufe der Jahre immer wieder Immobilien verschiedenster Art angeboten haben.
    \item Als Interessenten werden Leute bezeichnet, die die Immobilien kaufen wollen und die mit Nach- und Vornamen, vollständiger Adresse und Telefonnummer erfasst werden.
    \item Makler vereinbaren mit Interessenten Besuchstermine an einem bestimmten Datum zu einer bestimmten Uhrzeit.
    \item Wenn der Interessent an einer Immobilie interessiert ist, hinterlegt er ein Gebot, das in der Datenbank gespeichert wird, inklusive Information darüber, ob es angenommen wurde oder nicht.
\end{itemize}