%! Author = Luca
%! Date = 11.12.2022

Aus der Projektbeschreibung~\cite*{Auftrag} geht das in~\chapref{subsec:projektumfeld} beschriebene Projektumfeld hervor.
In~\chapref{subsec:projektidee} wird die Aufgabenstellung einmal verkürzt zusammengefasst.

\subsection{Projektumfeld}
\label{subsec:projektumfeld}
Das Projektumfeld für die Entwicklung einer relationalen Datenbank für die Schäfer \& Twachtmann Immobilien GbR beinhaltet die Immobilienbranche und das Maklerbüro selbst.
Das Unternehmen ist ein renommiertes Maklerbüro in Bremen und hat sich auf den Verkauf von Häusern und Wohnungen sowie die Suche nach Wunschimmobilien spezialisiert.
Es ist seit 25 Jahren im wachsenden Bremer Immobilienmarkt tätig und möchte nun seine Geschäftsprozesse neu strukturieren.
Dazu benötigt es eine relationale Datenbank, die von verschiedenen Programmen genutzt wird, darunter eine Terminplanungssoftware für die Makler und eine Verwaltungssoftware, die auf die Daten zugreift, um aktuelle Angebote auf der Unternehmenswebsite darzustellen und Exposés zu erstellen.
Die Datenbank wird spezifische Informationen zu Immobilien, Maklern, Eigentümern und Interessenten speichern.
Sie wird auch Informationen zu verschiedenen Immobilientypen enthalten und Gebote von Interessenten speichern, die angenommen oder abgelehnt wurden.
Das Projekt wird von Softwareentwicklern in der Entwicklungsabteilung der HiTec GmbH durchgeführt.

\subsection{Projektidee/ Kurzform der Aufgabenstellung}
\label{subsec:projektidee}
Das Projekt beinhaltet die Entwicklung einer relationalen Datenbank für die Schäfer \& Twachtmann Immobilien GbR, einem Maklerbüro in Bremen.
Die Datenbank soll verwendet werden, um die Geschäftsprozesse des Unternehmens zu strukturieren und umfasst Informationen über Immobilien, Makler, Eigentümer und Interessenten.
Die Datenbank wird von verschiedenen Programmen genutzt, darunter eine Terminplanungssoftware für die Makler und eine Verwaltungssoftware, die auf die Daten zugreift, um aktuelle Angebote auf der Unternehmenswebsite darzustellen und Exposés zu erstellen.
Die Datenbank wird spezifische Informationen zu jeder Immobilie speichern, wie Adresse, Wohnfläche, Anzahl der Zimmer, Preis, Einstellungsdatum, Verkaufsdatum und Baujahr.
Sie wird auch Informationen zu verschiedenen Immobilientypen enthalten, wie Einfamilienhäuser, Mehrfamilienhäuser, Reihenmittelhäuser und Reihenendhäuser.
Jede Immobilie wird von einem Makler betreut und die Datenbank wird Informationen zu den Maklern speichern, wie Nachnamen, Vornamen und Telefonnummern.
Die Eigentümer jeder Immobilie werden ebenfalls in der Datenbank gespeichert.
Die Schäfer \& Twachtmann Immobilien GbR hat auch Stammkunden, die über das Maklerbüro Immobilien verkauft haben, und die Datenbank wird Informationen zu Interessenten speichern, die Immobilien kaufen möchten.
Diese werden mit Nach- und Vornamen, Adresse und Telefonnummer gespeichert.
Die Makler vereinbaren mit Interessenten Besuchstermine an bestimmten Daten und Uhrzeiten und die Datenbank wird auch Informationen zu Geboten von Interessenten speichern, die angenommen oder abgelehnt wurden.

